\documentclass{article}

%----
%  Colin Tan
%  Basic setup for homework
%----
%----------------------------------------------------------------------------------------
%	PACKAGES AND OTHER DOCUMENT CONFIGURATIONS
%----------------------------------------------------------------------------------------

\usepackage{fancyhdr} % Required for custom headers
\usepackage{lastpage} % Required to determine the last page for the footer
\usepackage{extramarks} % Required for headers and footers
\usepackage{graphicx} % Required to insert images

\usepackage{listings} % listing codes

\usepackage{siunitx} % SI units
\usepackage{amsmath, amssymb} % Math

\usepackage{tikz} % Drawing graphs
\usepackage{pgfplots} % Drawing mathematical plots
\usepgfplotslibrary{fillbetween}
\pgfplotsset{compat=1.10} % pgf compatable version
\usepackage{float} % Flotation control

\usepackage{framed} % Framing answers
\usepackage{enumitem} % Customize enumeration style

\usepackage{multicol} % Required for columizing
\usepackage{caption} % For non-numbered captions
\usepackage{subcaption} % for caption of subfigures

\usepackage[us]{datetime} % Print date in US format

% Margins
\topmargin=-0.45in
\evensidemargin=0in
\oddsidemargin=0in
\textwidth=6.5in
\textheight=9.0in
\headsep=0.25in 

\linespread{1.1} % Line spacing

% Set up the header and footer
\pagestyle{fancy}
\lhead{\hmwkAuthorName} % Top left header
\chead{\hmwkClass\ (\hmwkClassInstructor\ \hmwkClassTime): \hmwkTitle} % Top center header
\rhead{\firstxmark} % Top right header
\lfoot{\lastxmark} % Bottom left footer
\cfoot{} % Bottom center footer
\rfoot{Page\ \thepage\ of\ \pageref{LastPage}} % Bottom right footer
\renewcommand\headrulewidth{0.4pt} % Size of the header rule
\renewcommand\footrulewidth{0.4pt} % Size of the footer rule

\setlength\parindent{0pt} % Removes all indentation from paragraphs

%----------------------------------------------------------------------------------------
%	DOCUMENT STRUCTURE COMMANDS
%----------------------------------------------------------------------------------------

% Header and footer for when a page split occurs within a problem environment
\newcommand{\enterProblemHeader}[1]{
	\nobreak\extramarks{#1}{#1 continued on next page\ldots}\nobreak
	\nobreak\extramarks{#1 (continued)}{#1 continued on next page\ldots}\nobreak
}

% Header and footer for when a page split occurs between problem environments
\newcommand{\exitProblemHeader}[1]{
	\nobreak\extramarks{#1 (continued)}{#1 continued on next page\ldots}\nobreak
	\nobreak\extramarks{#1}{}\nobreak
}

\setcounter{secnumdepth}{0} % Removes default section numbers
\newcounter{homeworkProblemCounter} % Creates a counter to keep track of the number of problems

\newcommand{\homeworkProblemName}{}
\newenvironment{homeworkProblem}[1][Problem \arabic{homeworkProblemCounter}]{ % Makes a new environment called homeworkProblem which takes 1 argument (custom name) but the default is "Problem #"
	\stepcounter{homeworkProblemCounter} % Increase counter for number of problems
	\renewcommand{\homeworkProblemName}{#1} % Assign \homeworkProblemName the name of the problem
	\section{\homeworkProblemName} % Make a section in the document with the custom problem count
	\enterProblemHeader{\homeworkProblemName} % Header and footer within the environment
}{
	\exitProblemHeader{\homeworkProblemName} % Header and footer after the environment
}

\newcommand{\problemAnswer}[1]{ % Defines the problem answer command with the content as the only argument
	\noindent\begin{oframed}
		#1
	\end{oframed}
}

\newcommand{\homeworkSectionName}{}
\newenvironment{homeworkSection}[1]{ % New environment for sections within homework problems, takes 1 argument - the name of the section
	\renewcommand{\homeworkSectionName}{#1} % Assign \homeworkSectionName to the name of the section from the environment argument
	\subsection{\homeworkSectionName} % Make a subsection with the custom name of the subsection
	\enterProblemHeader{\homeworkProblemName\ [\homeworkSectionName]} % Header and footer within the environment
}{
	\enterProblemHeader{\homeworkProblemName} % Header and footer after the environment
}

%----------------------------------------------------------------------------------------
%	TITLE PAGE
%----------------------------------------------------------------------------------------

\title{
\vspace{2in}
\textmd{\textbf{\hmwkClass:\ \hmwkTitle}}\\
\normalsize\vspace{0.1in}\small{Due\ on\ \hmwkDueDate}\\
\vspace{0.1in}\large{\textit{\hmwkClassInstructor\ \hmwkClassTime}}
\vspace{3in}
}

\author{\textbf{\hmwkAuthorName}}
\date{\today} % Insert date here if you want it to appear below your name

\usetikzlibrary{shapes.geometric, calc}
%\everymath{\displaystyle}
%----------------------------------------------------------------------------------------
%	NAME AND CLASS SECTION
%----------------------------------------------------------------------------------------

\newdate{DueDate}{18}{03}{2015} % Due date in {dd}{mm}{yyyy}
\newcommand{\hmwkTitle}{Homework\ 7} % Assignment title
\newcommand{\hmwkDueDate}{\dayofweekname{\getdateday{DueDate}}{\getdatemonth{DueDate}}{\getdateyear{DueDate}} \displaydate{DueDate}} % Due date
\newcommand{\hmwkClass}{PHYS\ 161} % Course/class
\newcommand{\hmwkClassTime}{11:00am} % Class/lecture time
\newcommand{\hmwkClassInstructor}{Professor Landee} % Teacher/lecturer
\newcommand{\hmwkAuthorName}{Zhuoming Tan} % Your name

%----------------------------------------------------------------------------------------

\begin{document}

\maketitle
\newpage
%----------------------------------------------------------------------------------------
%	TABLE OF CONTENTS
%----------------------------------------------------------------------------------------

%\setcounter{tocdepth}{1} % Uncomment this line if you don't want subsections listed in the ToC

%\newpage
%\tableofcontents
%\newpage

%----------------------------------------------------------------------------------------
%	PROBLEM 1
%----------------------------------------------------------------------------------------

% To have just one problem per page, simply put a \clearpage after each problem

\begin{homeworkProblem}
	(5.2) \textit{Maximum horizontal force}

	A charge $q_1$ is at rest at the origin, and a charge $q_2$ moves with speed $\beta c$ in the $x$ direction, along the line $z=b$. For what angle $\theta$ shown in Fig.~5.27 will the horizontal component of the force on $q_1$ be maximum? What is $\theta$ in the $\beta\approx1$ and $\beta\approx0$ limits?
	% Question

	\textbf{Solution}

	The radial field of the charge $q_2$ is
	\[
		E=\frac{q_2}{4\pi\epsilon_0 r^2}\frac{1-\beta^2}{\left(1-\beta^2\sin^2\theta\right)^{3/2}}
	\]
	where $r=b/\sin\theta$. So the $x$ component on the force on $q_1$ is
	\[
		F_x=\frac{q_1q_2\sin^2\theta}{4\pi\epsilon_0 b^2}\frac{1-\beta^2}{\left(1-\beta^2\sin^2\theta\right)^{3/2}}\cos\theta
	\]
	so
	\[
		{F_x}'=\frac{\left(\beta^2-1\right)q_1q_2\sin\theta\left(\left(\beta^2-3\right)\cos(2\theta)-\beta^2-1\right)}{\sqrt{2}\pi\beta^2\epsilon_0\left(\beta^2\cos(2\theta)-\beta^2+2\right)^{5/2}}
	\]
	Let ${F_x}'=0$, the solution is $\theta=k\pi$, or $\theta=\pm\frac{1}{2}\cos^{-1}\left(\frac{\beta^2+1}{\beta^2-3}\right)+k\pi$, where $k\in\mathbb{Z}$. Identify that in the range of $(0,\pi/2)$ the root $\theta=\frac{1}{2}\cos^{-1}\left(\frac{\beta^2+1}{\beta^2-3}\right)$ gives the maximum.

	When $\beta\approx1$, $\theta=\pi/2=\ang{90}$. When $\beta\approx0$, $\theta=\frac{1}{2}\cos^{-1}(-1/3)\approx\ang{54.7}$.
\end{homeworkProblem}

%----------------------------------------------------------------------------------------
%	PROBLEM 2
%----------------------------------------------------------------------------------------

\begin{homeworkProblem}
	(5.5) \textit{E from a line of moving charges}

	An essentially continuous stream of point charges moves with speed $v$ along the $x$ axis. The stream extends from $-\infty$ to $+\infty$. Let the charge density per unit length be $\lambda$, as measured in the lab frame. We know from using a cylindrical Gaussian surface that the electric field a distance $r$ from the $x$ axis is $E=\lambda/2\pi\epsilon_0r$. Derive this result again by using Eq.~(5.15) and integrating over all of the moving charges. You will want to use a computer or the integral table in Appendix K.
	% Question

	\textbf{Solution}

	Consider a small section $dq$ of the stream, making an angle $d\theta$ to the point $P$. The angle made between the perpendicular line from $P$ to the stream, $r$, and the line between $P$ and $dq$ is $\theta$.
	\[
		dq=\lambda\frac{r}{\cos^2\theta}\,d\theta
	\]
	The field is
	\[
		dE=\frac{dq}{4\pi\epsilon_0(r/\cos\theta)^2}\frac{1-\beta^2}{\left(1-\beta^2\cos^2\theta\right)^{3/2}}
	\]
	and by symmetry of the setup, only $y$ component of the field exists in the result. So
	\[
		E=\int_{-\pi/2}^{\pi/2}\frac{\lambda\frac{r}{\cos^2\theta}\,d\theta}{4\pi\epsilon_0(r/\cos\theta)^2}\frac{1-\beta^2}{\left(1-\beta^2\cos^2\theta\right)^{3/2}}\cos\theta
	\]
	Using a computer algebraic system, the result of this integral is exactly
	\[
		\frac{\lambda}{2\pi \epsilon_0 r}
	\]
\end{homeworkProblem}

%----------------------------------------------------------------------------------------
%	PROBLEM 3
%----------------------------------------------------------------------------------------

\begin{homeworkProblem}
	(5.8) \textit{Finding the magnetic field}
	
	Consider the second scenario in the example at the end of Section 5.8. Show that the total force in frame $F'$ equals the sum of the electric and magnetic forces, provided that there is a magnetic field pointing out of the page with magnitude $\gamma vE_2/c^2$.
	% Question

	\textbf{Solution}

	In the example, total force is $qE_2/\gamma$, and the electric force is $\gamma qE_2$, repulsive. So the attractive magnetic force must be of the magnitude
	\[
		\gamma qE_2-\frac{qE_2}{\gamma}=\gamma qE_2\left(1-\frac{1}{\gamma^2}\right)=\gamma qE_2\left(\frac{v^2}{c^2}\right)=qv\frac{\gamma v E_2}{c^2}
	\]
	Because of the formula $F_B=qvB$, $B=\gamma vE_2/c^2$.
\end{homeworkProblem}

%----------------------------------------------------------------------------------------
%	PROBLEM 4
%----------------------------------------------------------------------------------------

\begin{homeworkProblem}
	(5.9) \textit{``Twice'' the velocity}

	Suppose that the velocity of the test charge in Fig.~5.22 is chosen so that in its frame the electrons move backward with speed $v_0$.
	\begin{enumerate}[label=(\alph*)]
		\item Show that the $\beta$ associated with the test charge's velocity in the lab frame must be $\beta=2\beta_0/\left(1+{\beta_0}^2\right)$.
		\item Using length contraction, find the net charge density in the test-charge frame, and check that it agrees with Eq.~(5.24).
	\end{enumerate}
	% Question

	\textbf{Solution}

	\begin{equation}\tag{5.24}
		\lambda'=\gamma \beta \beta_0 \lambda_0
	\end{equation}
	\begin{enumerate}[label=(\alph*)]
		\begin{item}
			Test charge seeing electrons moving backwards at $v_0$ is equivalent to electrons seeing test charge moving forward at $v_0$. $\beta_0=v_0/c$. Velocity-addition formula states that
			\[
				\beta=\frac{v}{c}=\frac{\frac{\beta_0c+\beta_0c}{1+\frac{(\beta_0c)^2}{c^2}}}{c}=\frac{2\beta_0}{1+{\beta_0}^2}
			\]
		\end{item}
		\begin{item}
			\[
				\gamma=\frac{1}{\sqrt{1-\beta^2}}=\frac{1+{\beta_0}^2}{1-{\beta_0}^2}
			\]
			The test-charge is contracted, so density becomes $\lambda_0 \gamma$. Electrons move in opposite direction, but the speed is the same $v_0$. So density does not change. Total density then is
			\[
				\lambda'=\lambda_0(\gamma-1)=\lambda_0\frac{2{\beta_0}^2}{1-{\beta_0}^2}=\lambda_0\beta_0\frac{2\beta_0}{1+{\beta_0}^2}\frac{1+{\beta_0}^2}{1-{\beta_0}^2}=\gamma \beta \beta_0 \lambda_0
			\]
		\end{item}
	\end{enumerate}
\end{homeworkProblem}

%----------------------------------------------------------------------------------------
%	PROBLEM 5
%----------------------------------------------------------------------------------------

\begin{homeworkProblem}
	(5.12) \textit{Tilted sheet}

	Redo the ``Tilted sheet'' example in Section 5.5 in terms of a general $\gamma$ factor, to verify that Gauss's law holds for any choice of the relative speed of the two frames.
	% Question

	\textbf{Solution}

	From Gauss's law $E=\frac{\sigma}{2\epsilon_0}$. For a general $\gamma$, we have
	\[
		\sqrt{2}l \sigma=\sqrt{1+\left(\frac{1}{\gamma}\right)^2}l\sigma'\longrightarrow\sigma'=\frac{\sqrt{2}\gamma}{\sqrt{1+\gamma^2}}\sigma
	\]
	\[
		E_\parallel={E_\parallel}'=\frac{E}{\sqrt{2}}
	\]
	\[
		\gamma E_\perp={E_\perp}'=\frac{\gamma E}{\sqrt{2}}
	\]
	So in the triangle
	\[
		E'=\frac{E}{\sqrt{2}}\sqrt{1+\gamma^2}
	\]
	\[
		\frac{{E_\perp}'}{{E_\parallel}'}=\gamma\longrightarrow\theta=\arctan\gamma
	\]
	So
	\begin{align*}
		{E_n}'&=E'\cos\left(2\theta-\frac{\pi}{2}\right) \\
		&=\frac{E}{\sqrt{2}}\sqrt{1+\gamma^2}\cos\left(2\arctan\gamma-\frac{\pi}{2}\right) \\
		&=\frac{E}{\sqrt{2}}\sqrt{1+\gamma^2}\sin\left(2\arctan\gamma\right) \\
		&=\frac{E}{\sqrt{2}}\sqrt{1+\gamma^2}\frac{2\gamma}{1+\gamma^2} \\
		&=\frac{\sqrt{2}\gamma}{\sqrt{1+\gamma^2}}E \\
		&=\frac{\sqrt{2}\gamma}{\sqrt{1+\gamma^2}}\frac{\sigma}{2\epsilon_0} \\
		&=\frac{\sigma'}{2\epsilon_0}
	\end{align*}
	So Guass's law holds in frame $F'$.
\end{homeworkProblem}

%----------------------------------------------------------------------------------------
%	PROBLEM 6
%----------------------------------------------------------------------------------------

\begin{homeworkProblem}
	(5.15) \textit{Gauss's law for a moving charge}

	Verify that Gauss's law holds for the electric field in Eq.~(5.15). That is, verify that the flux of the field, through a sphere centered at the charge, is $q/\epsilon_0$. Of course, we used this fact in deriving Eq.~(5.15) in the first place, so we know that it must be true. But it can't hurt to double check. You'll want to use a computer or the integral table in Appendix K.
	% Question

	\textbf{Solution}

	\begin{equation}\tag{5.15}
		E'=\frac{Q}{4\pi\epsilon_0{r'}^2}\frac{1-\beta^2}{\left(1-\beta^2\sin^2\theta'\right)^{3/2}}
	\end{equation}
	So for charge $q$ in $\hat{\mathbf{r}}$ direction
	\begin{align*}
		\oint\mathbf{E}'\cdot\,d\mathbf{a}'&=\oint\frac{q}{4\pi\epsilon_0{r'}^2}\frac{1-\beta^2}{\left(1-\beta^2\sin^2\theta'\right)^{3/2}}{r'}^2\sin\theta'\,d\theta'\,d\phi' \\
		&=\frac{q}{4\pi\epsilon_0}\int_0^{2\pi}\int_0^\pi\frac{1-\beta^2}{\left(1-\beta^2\sin^2\theta'\right)^{3/2}}\sin\theta'\,d\theta'\,d\phi' \\
		&=\frac{q}{2\epsilon_0}\int_0^\pi\frac{1-\beta^2}{\left(1-\beta^2\sin^2\theta'\right)^{3/2}}\sin\theta'\,d\theta' \\
		&=\frac{q}{\epsilon_0}
	\end{align*}
\end{homeworkProblem}

%----------------------------------------------------------------------------------------
%	PROBLEM 7
%----------------------------------------------------------------------------------------

\begin{homeworkProblem}
	(5.22) \textit{Electron in an oscilloscope}

	The deflection plates in a high-voltage cathode ray oscilloscope are two rectangular plates, \SI{4}{\cm} long and \SI{1.5}{\cm} wide, and spaced \SI{0.8}{\cm} apart. There is a difference in potential of \SI{6000}{\volt} between the plates. An electron that has been accelerated through a potential difference of \SI{250}{\kilo\volt} enters this deflector from the left, moving parallel to the plates and halfway between them, initially. We want to find the position of the electron and its direction of motion when it leaves the deflecting field at the other end of the plates. We shall neglect the fringing field and assume the electric field between the plates is uniform right up to the end. The rest energy of the electron may be taken as \SI{500}{\kilo\electronvolt}.
	\begin{enumerate}[label=(\alph*)]
		\begin{item}
			First carry out the analysis in the lab frame by answering the following questions:
			\begin{itemize}
				\item What are the values of $\gamma$ and $\beta$?
				\item What is $p_x$ in units of $mc$?
				\item How long does the electron spend between the plates? (Neglect the change in horizontal velocity discussed in Exercise 5.25.)
				\item What is the transverse momentum component acquired, in units of $mc$?
				\item What is the transverse velocity at exit?
				\item What is the vertical position at exit?
				\item What is the direction of flight at exit?
			\end{itemize}
		\end{item}
		\item Now describe this whole process as it would appear in an inertial frame that moved with the electron at the moment it entered the deflecting region. What do the plates look like? What is the field between them? What happens to the electron in this coordinate system? Your main object in this exercise is to convince yourself that the two descriptions are completely consistent.
	\end{enumerate}
	% Question

	\textbf{Solution}

	The electron is accelerated through \SI{250}{\kilo\volt}, so it gains $K=\SI{250}{\kilo\electronvolt}$. The total energy becomes $E=K+mc^2=\SI{750}{\kilo\volt}=1.5mc^2$.
	\begin{enumerate}[label=(\alph*)]
		\begin{item}
			\begin{itemize}
				\item So $\gamma=1.5$. Then $\beta=\sqrt{1-(1/\gamma)^2}=0.7454$.
				\item $p_x=\gamma \beta mc$. $p_x/mc=\gamma \beta=1.1180$.
				\item $t=\frac{\SI{4}{\cm}}{\beta c}=\SI{1.7901e-10}{\s}$.
				\item $p_y=F_y t=eEt=eVt/s$. So $\frac{p_y}{mc}=\frac{eVt}{smc}=0.0805$.
				\item $p_y=\gamma m v_y\longrightarrow v_y=\SI{1.6089e7}{\m\per\s}$.
				\item $y=\frac{v_y t}{2}=\SI{1.4400e-3}{\m}$.
				\item $\theta=\arctan\frac{p_y}{p_x}=\ang{4.1183}$.
			\end{itemize}
		\end{item}
		\begin{item}
			In the frame where the electron is at rest, the planes are moving to $-x$ direction with $\beta=0.7454$. The contracted length of the plates is $d'=d/\gamma=\SI{2.6667}{\cm}$. The electron takes $t'=d'/\beta$ to be accelerated in $E'=\gamma E$.

			Therefore $p_y=eEt=eE'T'$, since $y$ direction is not change in the transformation.
		\end{item}
	\end{enumerate}
\end{homeworkProblem}

%----------------------------------------------------------------------------------------
%	PROBLEM 8
%----------------------------------------------------------------------------------------

\begin{homeworkProblem}
	(5.26) \textit{Charges in a wire}

	In Fig.~5.22 the relative spacing of the black and gray dots was designed to be consistent with $\gamma=1.2$ and $\beta_0=0.8$. Calculate ${\beta_0}'$. Find the value, as a fraction of $\lambda_0$, of the net charge density $\lambda'$ in the test-charge frame.
	% Question

	\textbf{Solution}
	Because
	\[
		\beta=\sqrt{1-\frac{1}{\gamma^2}}=0.552771
	\]
	\[
		{\beta_0}'=\frac{\beta_0-\beta}{1-\beta\beta_0}=0.443235
	\]
	\[
		\lambda'=\gamma\beta\beta_0\lambda_0=0.53066\lambda_0
	\]
\end{homeworkProblem}
%----------------------------------------------------------------------------------------

\end{document}
