\documentclass{article}

%----
%  Colin Tan
%  Basic setup for homework
%----
%----------------------------------------------------------------------------------------
%	PACKAGES AND OTHER DOCUMENT CONFIGURATIONS
%----------------------------------------------------------------------------------------

\usepackage{fancyhdr} % Required for custom headers
\usepackage{lastpage} % Required to determine the last page for the footer
\usepackage{extramarks} % Required for headers and footers
\usepackage{graphicx} % Required to insert images

\usepackage{listings} % listing codes

\usepackage{siunitx} % SI units
\usepackage{amsmath, amssymb} % Math

\usepackage{tikz} % Drawing graphs
\usepackage{pgfplots} % Drawing mathematical plots
\usepgfplotslibrary{fillbetween}
\pgfplotsset{compat=1.10} % pgf compatable version
\usepackage{float} % Flotation control

\usepackage{framed} % Framing answers
\usepackage{enumitem} % Customize enumeration style

\usepackage{multicol} % Required for columizing
\usepackage{caption} % For non-numbered captions
\usepackage{subcaption} % for caption of subfigures

\usepackage[us]{datetime} % Print date in US format

% Margins
\topmargin=-0.45in
\evensidemargin=0in
\oddsidemargin=0in
\textwidth=6.5in
\textheight=9.0in
\headsep=0.25in 

\linespread{1.1} % Line spacing

% Set up the header and footer
\pagestyle{fancy}
\lhead{\hmwkAuthorName} % Top left header
\chead{\hmwkClass\ (\hmwkClassInstructor\ \hmwkClassTime): \hmwkTitle} % Top center header
\rhead{\firstxmark} % Top right header
\lfoot{\lastxmark} % Bottom left footer
\cfoot{} % Bottom center footer
\rfoot{Page\ \thepage\ of\ \pageref{LastPage}} % Bottom right footer
\renewcommand\headrulewidth{0.4pt} % Size of the header rule
\renewcommand\footrulewidth{0.4pt} % Size of the footer rule

\setlength\parindent{0pt} % Removes all indentation from paragraphs

%----------------------------------------------------------------------------------------
%	DOCUMENT STRUCTURE COMMANDS
%----------------------------------------------------------------------------------------

% Header and footer for when a page split occurs within a problem environment
\newcommand{\enterProblemHeader}[1]{
	\nobreak\extramarks{#1}{#1 continued on next page\ldots}\nobreak
	\nobreak\extramarks{#1 (continued)}{#1 continued on next page\ldots}\nobreak
}

% Header and footer for when a page split occurs between problem environments
\newcommand{\exitProblemHeader}[1]{
	\nobreak\extramarks{#1 (continued)}{#1 continued on next page\ldots}\nobreak
	\nobreak\extramarks{#1}{}\nobreak
}

\setcounter{secnumdepth}{0} % Removes default section numbers
\newcounter{homeworkProblemCounter} % Creates a counter to keep track of the number of problems

\newcommand{\homeworkProblemName}{}
\newenvironment{homeworkProblem}[1][Problem \arabic{homeworkProblemCounter}]{ % Makes a new environment called homeworkProblem which takes 1 argument (custom name) but the default is "Problem #"
	\stepcounter{homeworkProblemCounter} % Increase counter for number of problems
	\renewcommand{\homeworkProblemName}{#1} % Assign \homeworkProblemName the name of the problem
	\section{\homeworkProblemName} % Make a section in the document with the custom problem count
	\enterProblemHeader{\homeworkProblemName} % Header and footer within the environment
}{
	\exitProblemHeader{\homeworkProblemName} % Header and footer after the environment
}

\newcommand{\problemAnswer}[1]{ % Defines the problem answer command with the content as the only argument
	\noindent\begin{oframed}
		#1
	\end{oframed}
}

\newcommand{\homeworkSectionName}{}
\newenvironment{homeworkSection}[1]{ % New environment for sections within homework problems, takes 1 argument - the name of the section
	\renewcommand{\homeworkSectionName}{#1} % Assign \homeworkSectionName to the name of the section from the environment argument
	\subsection{\homeworkSectionName} % Make a subsection with the custom name of the subsection
	\enterProblemHeader{\homeworkProblemName\ [\homeworkSectionName]} % Header and footer within the environment
}{
	\enterProblemHeader{\homeworkProblemName} % Header and footer after the environment
}

%----------------------------------------------------------------------------------------
%	TITLE PAGE
%----------------------------------------------------------------------------------------

\title{
\vspace{2in}
\textmd{\textbf{\hmwkClass:\ \hmwkTitle}}\\
\normalsize\vspace{0.1in}\small{Due\ on\ \hmwkDueDate}\\
\vspace{0.1in}\large{\textit{\hmwkClassInstructor\ \hmwkClassTime}}
\vspace{3in}
}

\author{\textbf{\hmwkAuthorName}}
\date{\today} % Insert date here if you want it to appear below your name

\usetikzlibrary{shapes.geometric, calc}
\DeclareSIUnit\c{\mathit{c}}
%\everymath{\displaystyle}
%----------------------------------------------------------------------------------------
%	NAME AND CLASS SECTION
%----------------------------------------------------------------------------------------

\newdate{DueDate}{22}{04}{2015} % Due date in {dd}{mm}{yyyy}
\newcommand{\hmwkTitle}{Homework\ 10} % Assignment title
\newcommand{\hmwkDueDate}{\dayofweekname{\getdateday{DueDate}}{\getdatemonth{DueDate}}{\getdateyear{DueDate}} \displaydate{DueDate}} % Due date
\newcommand{\hmwkClass}{PHYS\ 161} % Course/class
\newcommand{\hmwkClassTime}{11:00am} % Class/lecture time
\newcommand{\hmwkClassInstructor}{Professor Landee} % Teacher/lecturer
\newcommand{\hmwkAuthorName}{Zhuoming Tan} % Your name

%----------------------------------------------------------------------------------------

\begin{document}

\maketitle
\newpage
%----------------------------------------------------------------------------------------
%	TABLE OF CONTENTS
%----------------------------------------------------------------------------------------

%\setcounter{tocdepth}{1} % Uncomment this line if you don't want subsections listed in the ToC

%\newpage
%\tableofcontents
%\newpage

%----------------------------------------------------------------------------------------
%	PROBLEM 1
%----------------------------------------------------------------------------------------

% To have just one problem per page, simply put a \clearpage after each problem

\begin{homeworkProblem}
	(10.2) \textit{Force on a dielectric}\ldots (problem omitted)
	% Question

	\textbf{Solution}
	\begin{enumerate}[label = (\alph*)]
		\begin{item}
			\[
				C=C_1+C_2=\frac{\epsilon_0a(b+x\chi_e)}{s}
			\]
			So stored energy
			\[
				U=\frac{Q^2}{2C}=\frac{Q^2s}{2\epsilon_0a(b+x\chi_e)}
			\]
		\end{item}
		\begin{item}
			\[
				F=-\frac{dU}{dx}=\frac{Q^2s\chi_e}{2\epsilon_0a{(b+x\chi_e)}^2}
			\]
			And $U$ decreases as $x$ increases, so the force pulls the dielectric in.
		\end{item}
	\end{enumerate}
\end{homeworkProblem}

%----------------------------------------------------------------------------------------
%	PROBLEM 2
%----------------------------------------------------------------------------------------

\begin{homeworkProblem}
	(10.8) \textit{Force from an induced dipole}\ldots (problem omitted)
	% Question

	\textbf{Solution}

	For a charge $q$, its electric field will be $E=\frac{q}{4\pi\epsilon_0r^2}$. The induced dipole moment of the atom will be $p=\alpha E$, which points from the ion to the atom. The field at the ion due to the atom will be $\frac{2p}{4\pi\epsilon_0r^3}$. So force
	\[
		F=\frac{2pq}{4\pi\epsilon_0r^3}=\frac{2\alpha q^2}{{(4\pi\epsilon_0)}^2r^5}
	\]
	which would be attractive force. The potential
	\[
		U=-\int_\infty^r F\,dr'=-\frac{\alpha q^2}{2{(4\pi\epsilon_0)}^2r^4}
	\]
	From this equation one could solve for $r$ for the potential at \SI{4e-21}{\joule}, which would be \SI{9.4e-10}{\m}.
\end{homeworkProblem}

%----------------------------------------------------------------------------------------
%	PROBLEM 3
%----------------------------------------------------------------------------------------

\begin{homeworkProblem}
	(10.12) \textit{Boundary conditions on $D$}\ldots (problem omitted)
	% Question

	\textbf{Solution}

	$D_\perp$ is continuous, because $\nabla\cdot\mathbf{D}=\rho_\mathrm{free}$. No free charge so the divergence is zero. By divergence theorem, in flux would equal to out flux. So $D_{\perp,\mathrm{in}}A=D_{\perp,\mathrm{out}}A$, which means $D_\perp$ is continuous across the boundary.

	Because there is no discontinuity in $E_\parallel$, $D_\parallel$ would also be continuous. $\mathbf{D}=\epsilon_0\mathbf{E}+\mathbf{P}$, therefore discontinuity in $D_\parallel$ is the same as in $P_\parallel$. $\mathbf{P}=0$ outside, so $\Delta D_\parallel=-D_{\perp,\mathrm{in}}$.
\end{homeworkProblem}

%----------------------------------------------------------------------------------------
%	PROBLEM 4
%----------------------------------------------------------------------------------------

\begin{homeworkProblem}
	(10.14) \textit{Boundary conditions on $E$ and $B$}\ldots (problem omitted)
	% Question

	\textbf{Solution}

	The equations without free charges or currents are:
	\[
		\nabla\cdot\mathbf{D}=0,\quad\nabla\times\mathbf{E}=-\partial\mathbf{B}/\partial t;\quad
		\nabla\cdot\mathbf{B}=0,\quad\nabla\times\mathbf{B}=\mu_0\partial\mathbf{D}/\partial t;
	\]
	Our equations tell us that the net flux out of the volume is zero, so the perpendicular field on one side must equal the perpendicular field on the other. And for the parallel components, we can apply Stokes' theorem to the two ``curl'' equations, with the area chosen to be a thin rectangle, of vanishing area, spanning the surface. Our equations tell us that the line integral around the rectangle is zero, so the parallel field on one side must equal the parallel field on the other. So
	\[
		D_{1,\perp}=D_{2,\perp},\quad E_{1,\parallel}=E_{2,\parallel};\quad
		B_{1,\perp}=B_{2,\perp},\quad B_{1,\parallel}=B_{2,\parallel};
	\]
	Since $\mathbf{D}=\epsilon\mathbf{E}$ for a linear dielectric, $\epsilon_1E_{1,\perp}=\epsilon_2E_{2,\perp}$, therefore $E_{2,\perp}$ is discontinuous. But other components are, including the entire $\mathbf{B}$ field.
\end{homeworkProblem}

%----------------------------------------------------------------------------------------
%	PROBLEM 5
%----------------------------------------------------------------------------------------

\begin{homeworkProblem}
	(10.15) \textit{Charge densities on a capacitor}\ldots (problem omitted)
	% Question

	\textbf{Solution}

	We know that because the plates are a whole, $V$ must be the same for two regions. So $(C_1+C_2)V=Q$, where $C_1=\frac{\epsilon_0a(b-x)}{s}$, and $C_2=\frac{\kappa\epsilon_0ax}{s}$. Therefore
	\[
		E=\frac{V}{s}=\frac{Q}{\epsilon_0a(b-x)+\kappa\epsilon_0ax}=\frac{Q}{\epsilon_0a(b+x\chi_e)}
	\]
	So
	\[
		\sigma_1=\epsilon_0E=\frac{Q}{a(b+x\chi_e)}
	\]
	\[
		\sigma_2=\kappa\epsilon_0E=\frac{\kappa Q}{a(b+x\chi_e)}
	\]
	which are indeed decreasing as $x$ increases.
\end{homeworkProblem}

%----------------------------------------------------------------------------------------
%	PROBLEM 6
%----------------------------------------------------------------------------------------

\begin{homeworkProblem}
	(10.17) \textit{Maximum energy storage}\ldots (problem omitted)
	% Question

	\textbf{Solution}

	We know dielectric strength
	\[
		E_\mathrm{max}=\frac{\SI{14}{\kilo\volt}}{\SI{0.00254}{\cm}}=\SI{5.512e8}{\volt\per\meter}
	\]
	So
	\[
		U_\mathrm{max}=\int\frac{\epsilon{E_\mathrm{max}}^2}{2}\,dV=\frac{\epsilon{E_\mathrm{max}}^2\rho}{2}=\frac{\kappa\epsilon_0{E_\mathrm{max}}^2\rho}{2}=\SI{6.12e9}{\joule\per\kg}
	\]
	The mass of the dielectric is 75\% of the total mass. So
	\[
		mgh=0.75U_\mathrm{max}m
	\]
	and
	\[
		h=\frac{3U_\mathrm{max}}{4g}=\SI{4.68e8}{\meter}
	\]
\end{homeworkProblem}

%----------------------------------------------------------------------------------------
%	PROBLEM 7
%----------------------------------------------------------------------------------------

\begin{homeworkProblem}
	(10.24) \textit{Field lines of a dipole}

	A field line in the dipole field is described in polar coordinates by the very simple equation $r=r_0\sin^2\theta$, in which $r_0$ is the radius at which the field line passes through the equatorial plane of the dipole. Show that this is true by demonstrating that at any point on that curve the tangent has the same direction as the dipole field.
	% Question

	\textbf{Solution}

	The equatorial plane is the plane bisecting the dipole. The equation $r=r_0\sin^2\theta\hat{r}$, which apperently should be pointing at the $\hat{r}$ direction, creates dumbell shapes around the dipole. So the tangent direction of the equation
	\[
		\frac{\partial r}{\partial\theta}=r_0\left(2\sin\theta\cos\theta\hat{r}+\sin^2\theta\frac{\partial\hat{r}}{\partial\theta}\right)=r_0\sin\theta\left(2\cos\theta\hat{r}+\sin\theta\hat{\theta}\right)
	\]
	And the field far away from the dipole is
	\[
		\mathbf{E}=\frac{P}{4\pi\epsilon_0r^2}\left(2\cos\theta\hat{r}+\sin\theta\hat{\theta}\right)
	\]
	So they are pointing at the same direction.
\end{homeworkProblem}

%----------------------------------------------------------------------------------------
%	PROBLEM 8
%----------------------------------------------------------------------------------------

\begin{homeworkProblem}
	(10.37) \textit{$E$ at the center of a polarized sphere}

	If you don't trust the $\mathbf{E}=-\mathbf{P}/3\epsilon_0$ result we obtained in Section 10.9 for the field inside a uniformly polarized sphere, you will find it more believable if you check it in a special case. By direct integration of the contributions from the $\sigma=P\cos\theta$ surface charge density, show that the field at the center is directed downward (assuming $\mathbf{P}$ points upward) with magnitude $P/3\epsilon_0$.
	% Question

	\textbf{Solution}

	Consider a ring on the spherical sheet. An element
	\begin{align*}
		dq&=\sigma\ da \\
		&=P\cos\theta R\sin\theta R\ d\theta\ d\phi \\
		&=PR^2\sin\theta\cos\theta\ d\theta\ d\phi \\
		&=2\pi PR^2\sin\theta\cos\theta\ d\theta
	\end{align*}
	So the vertical component of the field is
	\[
		dE_z=\frac{1}{4\pi\epsilon_0R^2}dq\cdot\cos\theta=\frac{P\sin\theta\cos^2\theta\ d\theta}{2\epsilon_0}
	\]
	The field is then
	\[
		E_z=\int dE_z=\frac{P}{2\epsilon_0}\int_0^\pi\sin\theta\cos^2\theta\ d\theta=\frac{P}{2\epsilon_0}\frac{2}{3}=\frac{P}{3\epsilon_0}
	\]
\end{homeworkProblem}

%----------------------------------------------------------------------------------------
%	PROBLEM 9
%----------------------------------------------------------------------------------------

\begin{homeworkProblem}
	(10.41) \textit{Discontinuity in $D_\parallel$}

	Consider the polarized sphere from Section 10.9. Using the forms of the internal and external electric fields, find the discontinuity in $D_\parallel$ across the surface of the sphere, as a function of $\theta$.
	% Question

	\textbf{Solution}
	\[
		\mathbf{D}=\epsilon\mathbf{E}+\mathbf{P}
	\]
	For the inside $\mathbf{D}=\frac{2}{3}\mathbf{P}$, so the parallel component is
	\[
		D_{\parallel,\mathrm{in}}=\frac{2P\sin\theta}{3}\hat{\theta}
	\]
	and the perpendicular component is
	\[
		D_{\perp,\mathrm{in}}=\frac{2P\cos\theta}{3}\hat{r}
	\]
	The field outside is a dipole field
	\[
		\mathbf{E}_\mathrm{out}=\frac{P}{3\epsilon_0}\left(2\cos\theta\hat{r}+\sin\theta\hat{theta}\right)
	\]
	and $\mathbf{D}_\mathrm{out}=\epsilon_0\mathbf{E}_\mathrm{out}$, so
	\[
		D_{\parallel,\mathrm{out}}=\frac{P\sin\theta}{3}\hat{\theta}
	\]
	So the discontinuity is $\frac{P\sin\theta}{3}$.
\end{homeworkProblem}

%----------------------------------------------------------------------------------------
%	PROBLEM 10
%----------------------------------------------------------------------------------------

\begin{homeworkProblem}
	(10.42) \textit{Energy density in a dielectric}

	By considering how the introduction of a dielectric changes the energy stored in a capacitor, show that the correct expression for the energy density in a dielectric must be $\epsilon E^2/2$. Then compare the energy stored in the electric field with that stored in the magnetic field in the wave studied in Section 10.15.
	% Question

	\textbf{Solution}

	The energy stored in a capacitor is $\frac{1}{2}CV^2$, where in a parallel plate $C=\frac{\kappa\epsilon_0 A}{d}$, and $E=V/d$. So
	\[
		U=\frac{1}{2}CV^2=\frac{1}{2}\frac{\kappa\epsilon_0 A}{d}E^2d^2=\frac{1}{2}\epsilon AE^2d
	\]
	The energy density is $U$ divided by volume $Ad$, which is exactly $\epsilon E^2/2$.

	For the problem, the energy density of the electric field is
	\[
		U_E=\frac{1}{2}\epsilon E^2
	\]
	and of the magnetic field is
	\[
		U_B=\frac{1}{2}\frac{B^2}{\mu_0}
	\]
	The average value of the two fields are simply $\langle E\rangle=E_0/2$, and $\langle B\rangle=B_0/2$. Because in this field $E_0=\pm\frac{B_0}{\sqrt{\mu_0\epsilon}}$,
	\[
		U_E=\frac{1}{8}\epsilon{\left(\frac{B_0}{\sqrt{\mu_0\epsilon}}\right)}^2
	\]
	\[
		U_B=\frac{1}{8}\frac{{B_0}^2}{\mu_0}
	\]
	So the energy stored in each of the two fields turn out to be equal.
\end{homeworkProblem}
%----------------------------------------------------------------------------------------

\end{document}
