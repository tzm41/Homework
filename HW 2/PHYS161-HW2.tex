\documentclass{article}

%----
%  Colin Tan
%  Basic setup for homework
%----
%----------------------------------------------------------------------------------------
%	PACKAGES AND OTHER DOCUMENT CONFIGURATIONS
%----------------------------------------------------------------------------------------

\usepackage{fancyhdr} % Required for custom headers
\usepackage{lastpage} % Required to determine the last page for the footer
\usepackage{extramarks} % Required for headers and footers
\usepackage{graphicx} % Required to insert images

\usepackage{listings} % listing codes

\usepackage[per-mode=symbol]{siunitx} % SI units
\usepackage{amsmath, amssymb} % Math

\usepackage{tikz} % Drawing graphs
\usepackage{pgfplots} % Drawing mathematical plots
\usepgfplotslibrary{fillbetween}
\pgfplotsset{compat=1.10} % pgf compatable version
\usepackage{float} % Flotation control

\usepackage{framed} % Framing answers
\usepackage{enumitem} % Customize enumeration style

\usepackage{multicol} % Required for columizing
\usepackage{caption} % For non-numbered captions
\usepackage{subcaption} % for caption of subfigures

\usepackage[us]{datetime} % Print date in US format

% Margins
\topmargin=-0.45in
\evensidemargin=0in
\oddsidemargin=0in
\textwidth=6.5in
\textheight=9.0in
\headsep=0.25in 

\linespread{1.1} % Line spacing

% Set up the header and footer
\pagestyle{fancy}
\lhead{\hmwkAuthorName} % Top left header
\chead{\hmwkClass\ (\hmwkClassInstructor\ \hmwkClassTime): \hmwkTitle} % Top center header
\rhead{\firstxmark} % Top right header
\lfoot{\lastxmark} % Bottom left footer
\cfoot{} % Bottom center footer
\rfoot{Page\ \thepage\ of\ \pageref{LastPage}} % Bottom right footer
\renewcommand\headrulewidth{0.4pt} % Size of the header rule
\renewcommand\footrulewidth{0.4pt} % Size of the footer rule

\setlength\parindent{0pt} % Removes all indentation from paragraphs

%----------------------------------------------------------------------------------------
%	DOCUMENT STRUCTURE COMMANDS
%----------------------------------------------------------------------------------------

% Header and footer for when a page split occurs within a problem environment
\newcommand{\enterProblemHeader}[1]{
	\nobreak\extramarks{#1}{#1 continued on next page\ldots}\nobreak
	\nobreak\extramarks{#1 (continued)}{#1 continued on next page\ldots}\nobreak
}

% Header and footer for when a page split occurs between problem environments
\newcommand{\exitProblemHeader}[1]{
	\nobreak\extramarks{#1 (continued)}{#1 continued on next page\ldots}\nobreak
	\nobreak\extramarks{#1}{}\nobreak
}

\setcounter{secnumdepth}{0} % Removes default section numbers
\newcounter{homeworkProblemCounter} % Creates a counter to keep track of the number of problems

\newcommand{\homeworkProblemName}{}
\newenvironment{homeworkProblem}[1][Problem \arabic{homeworkProblemCounter}]{ % Makes a new environment called homeworkProblem which takes 1 argument (custom name) but the default is "Problem #"
	\stepcounter{homeworkProblemCounter} % Increase counter for number of problems
	\renewcommand{\homeworkProblemName}{#1} % Assign \homeworkProblemName the name of the problem
	\section{\homeworkProblemName} % Make a section in the document with the custom problem count
	\enterProblemHeader{\homeworkProblemName} % Header and footer within the environment
}{
	\exitProblemHeader{\homeworkProblemName} % Header and footer after the environment
}

\newcommand{\problemAnswer}[1]{ % Defines the problem answer command with the content as the only argument
	\noindent\begin{oframed}
		#1
	\end{oframed}
}

\newcommand{\homeworkSectionName}{}
\newenvironment{homeworkSection}[1]{ % New environment for sections within homework problems, takes 1 argument - the name of the section
	\renewcommand{\homeworkSectionName}{#1} % Assign \homeworkSectionName to the name of the section from the environment argument
	\subsection{\homeworkSectionName} % Make a subsection with the custom name of the subsection
	\enterProblemHeader{\homeworkProblemName\ [\homeworkSectionName]} % Header and footer within the environment
}{
	\enterProblemHeader{\homeworkProblemName} % Header and footer after the environment
}

%----------------------------------------------------------------------------------------
%	TITLE PAGE
%----------------------------------------------------------------------------------------

\title{
\vspace{2in}
\textmd{\textbf{\hmwkClass:\ \hmwkTitle}}\\
\normalsize\vspace{0.1in}\small{Due\ on\ \hmwkDueDate}\\
\vspace{0.1in}\large{\textit{\hmwkClassInstructor\ \hmwkClassTime}}
\vspace{3in}
}

\author{\textbf{\hmwkAuthorName}}
\date{\today} % Insert date here if you want it to appear below your name

\usetikzlibrary{shapes.geometric, calc}
%\everymath{\displaystyle}
%----------------------------------------------------------------------------------------
%	NAME AND CLASS SECTION
%----------------------------------------------------------------------------------------

\newdate{DueDate}{28}{01}{2015} % Due date in {dd}{mm}{yyyy}
\newcommand{\hmwkTitle}{Homework\ 2} % Assignment title
\newcommand{\hmwkDueDate}{\dayofweekname{\getdateday{DueDate}}{\getdatemonth{DueDate}}{\getdateyear{DueDate}} \displaydate{DueDate}} % Due date
\newcommand{\hmwkClass}{PHYS\ 161} % Course/class
\newcommand{\hmwkClassTime}{11:00am} % Class/lecture time
\newcommand{\hmwkClassInstructor}{Professor Landee} % Teacher/lecturer
\newcommand{\hmwkAuthorName}{Zhuoming Tan} % Your name

%----------------------------------------------------------------------------------------

\begin{document}

\maketitle
\newpage
%----------------------------------------------------------------------------------------
%	TABLE OF CONTENTS
%----------------------------------------------------------------------------------------

%\setcounter{tocdepth}{1} % Uncomment this line if you don't want subsections listed in the ToC

%\newpage
%\tableofcontents
%\newpage

%----------------------------------------------------------------------------------------
%	PROBLEM 1
%----------------------------------------------------------------------------------------

% To have just one problem per page, simply put a \clearpage after each problem

\begin{homeworkProblem}
	(II-1) Find a unit vector $\hat{\mathbf{n}}$ normal to each of the following surfaces.
	\begin{multicols}{2}
		\begin{enumerate}[label=(\alph*)]
			\item $z=2-x-y$
			\item $z=(x^2+y^2)^{1/2}$
			\item $z=(1-x^2)^{1/2}$
			\item $z=x^2+y^2$
			\item $z=(1-x^2/a^2-y^2/a^2)^{1/2}$
		\end{enumerate}
	\end{multicols}
	% Question

	\problemAnswer{
		Using the principle ideal of (II-4),
		\begin{enumerate}[label=(\alph*)]
			\begin{item}
				$f(x,y,z)=x+y+z-2=0$. So $\mathbf{n}=\nabla f=
				\begin{bmatrix}
					1 \\
					1 \\
					1 \\
				\end{bmatrix}$. $\hat{\mathbf{n}}=
				\begin{bmatrix}
					1/3^{1/3} \\
					1/3^{1/3} \\
					1/3^{1/3} \\
				\end{bmatrix}$.
			\end{item}
			\begin{item}
				For $z=(x^2+y^2)^{1/2}$, $\mathbf{n}=
				\begin{bmatrix}
					\frac{\partial}{\partial x}(x^2+y^2)^{1/2} \\
					\frac{\partial}{\partial y}(x^2+y^2)^{1/2} \\
					-1
				\end{bmatrix}=
				\begin{bmatrix}
					x(x^2+y^2)^{-1/2} \\
					y(x^2+y^2)^{-1/2} \\
					-1
				\end{bmatrix}$. $\hat{\mathbf{n}}=
				\begin{bmatrix}
					x((x^2+y^2)/2)^{-1/2} \\
					y((x^2+y^2)/2)^{-1/2} \\
					-1/\sqrt{2}
				\end{bmatrix}$.
			\end{item}
			\begin{item}
				$\displaystyle\hat{\mathbf{n}}=\frac{\hat{\mathbf{i}}\frac{\partial f}{\partial x}+\hat{\mathbf{j}}\frac{\partial f}{\partial y}-\hat{\mathbf{k}}}{\sqrt{1+\left(\frac{\partial f}{\partial x}\right)^2+\left(\frac{\partial f}{\partial y}\right)^2}}=\frac{-x(1-x^2)^{-1/2}\hat{\mathbf{i}}-\hat{\mathbf{k}}}{\sqrt{\frac{1}{1-x^2}}}=-x\hat{\mathbf{i}}-(1-x^2)^{1/2}\hat{\mathbf{k}}=-x\hat{\mathbf{i}}-z\hat{\mathbf{k}}$
			\end{item}
			\begin{item}
				$\displaystyle\hat{\mathbf{n}}=\frac{\hat{\mathbf{i}}\frac{\partial f}{\partial x}+\hat{\mathbf{j}}\frac{\partial f}{\partial y}-\hat{\mathbf{k}}}{\sqrt{1+\left(\frac{\partial f}{\partial x}\right)^2+\left(\frac{\partial f}{\partial y}\right)^2}}=\frac{2x\hat{\mathbf{i}}+2y\hat{\mathbf{j}}-\hat{\mathbf{k}}}{\sqrt{1+4x^2+4y^2}}=\frac{2x\hat{\mathbf{i}}+2y\hat{\mathbf{j}}-\hat{\mathbf{k}}}{\sqrt{1+4z}}$
			\end{item}
			\begin{item}
				\begin{align*}
					\displaystyle\hat{\mathbf{n}}=&\frac{\hat{\mathbf{i}}\frac{\partial f}{\partial x}+\hat{\mathbf{j}}\frac{\partial f}{\partial y}-\hat{\mathbf{k}}}{\sqrt{1+\left(\frac{\partial f}{\partial x}\right)^2+\left(\frac{\partial f}{\partial y}\right)^2}} \\
					=&\frac{-x/a^2(1-x^2/a^2-y^2/a^2)^{-1/2}\hat{\mathbf{i}}-y/a^2(1-x^2/a^2-y^2/a^2)^{-1/2}\hat{\mathbf{j}}-\hat{\mathbf{k}}}{\sqrt{1+(4x^2+4y^2)/(a^4(1-x^2/a^2-y^2/a^2))}} \\
					=&\frac{-x\hat{\mathbf{i}}-y\hat{\mathbf{j}}-a^2z\hat{\mathbf{k}}}{a\sqrt{a^2+4(x^2+y^2)}}
				\end{align*}
			\end{item}
		\end{enumerate}
	}
\end{homeworkProblem}

%----------------------------------------------------------------------------------------
%	PROBLEM 2
%----------------------------------------------------------------------------------------

\begin{homeworkProblem}
	(II-2)
	\begin{enumerate}[label=(\alph*)]
		\begin{item}
			Show that the unit vector normal to the plane
			\[
				ax+by+cz=d
			\]
			is given by
			\[
				\hat{\mathbf{n}}=\pm(\hat{\mathbf{i}}a+\hat{\mathbf{j}}b+\hat{\mathbf{k}}c)/(a^2+b^2+c^2)^{1/2}
			\]
		\end{item}
		\item Explain in geometric terms why this expression for $\hat{\mathbf{n}}$ is independent of the constant $d$.
	\end{enumerate}
	% Question

	\problemAnswer{
		\begin{enumerate}[label=(\alph*)]
			\begin{item}
				As indicated in book, for $f(x,y,z)=ax+by+cz+d$, $\displaystyle\hat{\mathbf{n}}=\pm\frac{\hat{\mathbf{i}}\frac{\partial f}{\partial x}+\hat{\mathbf{j}}\frac{\partial f}{\partial y}+\hat{\mathbf{k}}\frac{\partial f}{\partial z}}{\sqrt{\left(\frac{\partial f}{\partial x}\right)^2+\left(\frac{\partial f}{\partial y}\right)^2+\left(\frac{\partial f}{\partial z}\right)^2}}=\pm\frac{\hat{\mathbf{i}}a+\hat{\mathbf{j}}b+\hat{\mathbf{k}}c}{(a^2+b^2+c^2)^{1/2}}$.
			\end{item}
			\item $d$ in the equation could be imagined as shifting the surface up or down along $z$ axis, which do not change the shape of the surface. Therefore the unit normal vector does not change at all.
		\end{enumerate}
	}
\end{homeworkProblem}

%----------------------------------------------------------------------------------------
%	PROBLEM 3
%----------------------------------------------------------------------------------------

\begin{homeworkProblem}
	(II-4) In each of the following use Equation II-12 to evaluate the surface integral $\iint_S G(x,y,z)\,dS$.
	\begin{enumerate}[label=(\alph*)]
		\item $$G(x,y,z)=z$$ where $S$ is the portion of the plane $x+y+z=1$ in the first octant.
		\item $$G(x,y,z)=\frac{1}{1+4(x^2+y^2)}$$ where $S$ is the portion of the paraboloid $z=x^2+y^2$ between $z=0$ and $z=1$.
		\item $$G(x,y,z)=(1-x^2-y^2)^{3/2}$$ where $S$ is the hemisphere $z=(1-x^2-y^2)^{1/2}$.
	\end{enumerate}
	% Question

	\problemAnswer{
		\begin{equation}\tag{II-12}
			\iint_S G(x,y,z)\,dS=\iint_R G[x,y,f(x,y)]\cdot\sqrt{1+\left(\frac{\partial f}{\partial x}\right)^2+\left(\frac{\partial f}{\partial y}\right)^2}\,dx\,dy
		\end{equation}
		\begin{enumerate}[label=(\alph*)]
			\begin{item}
				So $z=f(x,y)=1-x-y$. This gives $\frac{\partial f}{\partial x}=-1$, and $\frac{\partial f}{\partial y}=-1$. So $\sqrt{1+\left(\frac{\partial f}{\partial x}\right)^2+\left(\frac{\partial f}{\partial y}\right)^2}=\sqrt{3}$. The region $R$ becomes the area in $xy$-plane in the first quardrant that $x+y<1$, a triangle. So
				\[
					\iint_S z\,dS=\sqrt{3}\iint_R(1-x-y)\,dx\,dy=\sqrt{3}\int_0^{1-x}\int_0^1(1-x-y)\,dx\,dy=\frac{\sqrt{3}}{6}
				\]
			\end{item}
			\begin{item}
				$z=f(x,y)=x^2+y^2$. This gives $\frac{\partial f}{\partial x}=2x$, and $\frac{\partial f}{\partial y}=2y$. So $\sqrt{1+\left(\frac{\partial f}{\partial x}\right)^2+\left(\frac{\partial f}{\partial y}\right)^2}=\sqrt{1+4(x^2+y^2)}$. The region $R$ becomes the circle in $xy$-plane with $x^2+y^2=1$.
				\[
					\iint_S z\,dS=\iint_R\frac{1}{\sqrt{1+4(x^2+y^2)}}\,dx\,dy=\int_0^1\int_0^{2\pi}\frac{1}{\sqrt{1+4r^2}}r\,d\theta\,dr=\frac{(\sqrt{5}-1)\pi}{2}
				\]
			\end{item}
			\begin{item}
				$\frac{\partial f}{\partial x}=-x(1-x^2-y^2)^{-1/2}$, and $\frac{\partial f}{\partial y}=-y(1-x^2-y^2)^{-1/2}$. $\sqrt{1+\left(\frac{\partial f}{\partial x}\right)^2+\left(\frac{\partial f}{\partial y}\right)^2}=\sqrt{1+\frac{x^2+y^2}{1-x^2-y^2}}=(1-x^2-y^2)^{-1/2}$. The region $R$ is the circle in $xy$-plane with $x^2+y^2=1$.
				\[
					\iint_S z\,dS=\iint_R(1-x^2-y^2)\,dx\,dy=\int_0^1\int_0^{2\pi}(1-r^2)r\,d\theta\,dr=\frac{\pi}{2}
				\]
			\end{item}
		\end{enumerate}
	}
\end{homeworkProblem}

%----------------------------------------------------------------------------------------
%	PROBLEM 4
%----------------------------------------------------------------------------------------

\begin{homeworkProblem}
	(II-5) In each of the following... (problem omitted)
	% Question

	\problemAnswer{
		\begin{equation}\tag{II-13}
			\iint_S\mathbf{F}\cdot\hat{\mathbf{n}}\,dS=\iint_R\left\{-F_x[x,y,f(x,y)]\frac{\partial f}{\partial x}-F_y[x,y,f(x,y)]\frac{\partial f}{\partial y}+F_z[x,y,f(x,y)]\right\}\,dx\,dy
		\end{equation}
		\begin{enumerate}[label=(\alph*)]
			\begin{item}
				$z=f(x,y)=1-x/2-y/2$. $\frac{\partial f}{\partial x}=-1/2$, $\frac{\partial f}{\partial y}=-1/2$. $F_x=x$, $F_y=0$, $F_z=-z$. $R$ is the triangle region in the first quadrant of the $xy$-plane. So
				\[
					\iint_S\mathbf{F}\cdot\hat{\mathbf{n}}\,dS=\int_0^{2-x}\int_0^2(x+y/2-1)\,dx\,dy=0
				\]
			\end{item}
			\begin{item}
				$\frac{\partial f}{\partial x}=-x(a^2-x^2-y^2)^{-1/2}$, $\frac{\partial f}{\partial y}=-y(a^2-x^2-y^2)^{-1/2}$.
				\[
					\iint_S\mathbf{F}\cdot\hat{\mathbf{n}}\,dS=\iint_R\left(\frac{x^2+y^2}{(a^2-x^2-y^2)^{1/2}}+(a^2-x^2-y^2)^{1/2}\right)\,dx\,dy=\iint_R\frac{a^2}{(a^2-x^2-y^2)^{1/2}}\,dx\,dy
				\]
				With a change in varibles,
				\[
					\iint_R\frac{a^2}{(a^2-x^2-y^2)^{1/2}}\,dx\,dy=\int_0^a\int_0^{2\pi}\frac{a^2}{(a^2-r^2)^{1/2}}r\,d\theta\,dr=2\pi a^3
				\]
			\end{item}
			\begin{item}
				$\frac{\partial f}{\partial x}=-2x$, $\frac{\partial f}{\partial y}=-2y$. $R$ again becomes the circle $x^2+y^2=1$ on the $xy$-plane. The integral then becomes
				\[
					\iint_R(2y^2+1)\,dx\,dy=4\int_0^{\sqrt{1-x^2}}\int_0^1(2y^2+1)\,dx\,dy=\frac{3\pi}{2}
				\]
			\end{item}
		\end{enumerate}
	}
\end{homeworkProblem}

%----------------------------------------------------------------------------------------
%	PROBLEM 5
%----------------------------------------------------------------------------------------

\begin{homeworkProblem}
	(II-6) The distribution of mass... (problem omitted)
	% Question

	\problemAnswer{
		Total mass of the shell is
		\[
			\iint\sigma(x,y,z)\,ds=\iint\sigma(x,y)\sqrt{1+\left(\frac{\partial f}{\partial x}\right)^2+\left(\frac{\partial f}{\partial y}\right)^2}\,dx\,dy=\frac{1}{R}\iint\frac{\sigma_0(x^2+y^2)}{\sqrt{R^2-x^2-y^2}}
		\]
		A change of variables gives
		\[
			\frac{\sigma_0}{R}\int_0^R\int_0^{2\pi}\frac{r^3}{\sqrt{R^2-r^2}}\,d\theta\,dr=\frac{4\pi R^2\sigma_0}{3}
		\]
	}
\end{homeworkProblem}

%----------------------------------------------------------------------------------------
%	PROBLEM 6
%----------------------------------------------------------------------------------------

\begin{homeworkProblem}
	(II-10) It sometimes happens that surface... (problem omitted)
	% Question

	\problemAnswer{
	(II-30)
		\begin{enumerate}[label=(\alph*)]
			\item Take the $S$ square in $xy$-plane as example. The integral becomes $\iint_Sz\,dS$, which obviously equals 0. The same conclusion applies to the other two squares. So the integral is 0.
			\begin{item}
				The top and bottom contribute 0 to the integral. We only have to calculate the side surface. From
				\begin{equation}\tag{II-30}
					\iint_S\mathbf{F}\cdot\hat{\mathbf{n}}\,dS=\iiint_V\nabla\cdot\mathbf{F}\,dV
				\end{equation}
				$\nabla\cdot\mathbf{F}=\ln(x^2+y^2)+\frac{2x^2}{x^2+y^2}+\ln(x^2+y^2)+\frac{2y^2}{x^2+y^2}=2\ln(x^2+y^2)+2$. So
				\[
					\iint_S\mathbf{F}\cdot\hat{\mathbf{n}}\,dS=\iiint_V(2\ln(x^2+y^2)+2)\,dV
				\]
				Change the variables into cylindrical coordinates.
				\[
					\iiint_V(2\ln(x^2+y^2)+2)\,dV=\int_0^h\int_0^{2\pi}\int_0^R(2\ln(r^2)+2)r\,dr\,d\phi\,dz=4h\pi R^2\ln R
				\]
				It is now obvious that the integral is actually $V\ln R^2$, where $V$ is the volume of the cylinder and $R^2=x^2+y^2$.
			\end{item}
			\begin{item}
				For the sphere with radius $R$, $\mathbf{F}=(\mathbf{i}x+\mathbf{j}y+\mathbf{k}z)e^{-R^2}$. $\nabla\cdot\mathbf{F}=3e^{-R^2}$. Therefore $\iint_S\mathbf{F}\cdot\hat{\mathbf{n}}\,dS=\iiint_V\nabla\cdot\mathbf{F}\,dV=e^{-R^2}\iiint_V\,dV=3e^{-R^2}\times\frac{4}{3}\pi R^3=4\pi R^3e^{-R^2}$.
			\end{item}
			\item Only two surfaces parallel to the $yz$-plane has contribution to the integral. $\nabla\cdot\mathbf{F}=E'(x)$. Therefore $\iint_S\mathbf{F}\cdot\hat{\mathbf{n}}\,dS=\iiint_V\nabla\cdot\mathbf{F}\,dV=b^2\int_0^bE'(x)\,dx=b^2[E(b)-E(0)]$.
		\end{enumerate}
	}
\end{homeworkProblem}

%----------------------------------------------------------------------------------------
%	PROBLEM 7
%----------------------------------------------------------------------------------------

\begin{homeworkProblem}
	(II-14) Calculate the divergence... (problem omitted)
	% Question

	\problemAnswer{
		This is rather simple.
		\begin{multicols}{2}
			\begin{enumerate}[label=(\alph*)]
				\item $2x+2y+2z$
				\item 0
				\item $-e^{-x}-e^{-y}-e^{-z}$
				\item $2z$
				\item $\frac{-y}{x^2+y^2}$
				\item 0
				\item 3
				\item 0
			\end{enumerate}
		\end{multicols}
	}
\end{homeworkProblem}

%----------------------------------------------------------------------------------------
%	PROBLEM 8
%----------------------------------------------------------------------------------------

\begin{homeworkProblem}
	(II-15) (a) Calculate $\iint_S\mathbf{F}\cdot\hat{\mathbf{n}}\,dS$ for... (problem omitted)
	% Question

	\problemAnswer{
		\begin{enumerate}[label=(\alph*)]
			\begin{item}
				\begin{align*}
					\iint_S\mathbf{F}\cdot\hat{\mathbf{n}}\,dS&=\iiint_V\nabla\cdot\mathbf{F}\,dV \\
					&=2\iiint_V(x+y+z)\,dV \\
					&=2\int_{x_0-s/2}^{x_0+s/2}\int_{y_0-s/2}^{y_0+s/2}\int_{z_0-s/2}^{z_0+s/2}(x+y+z)\,dz\,dy\,dx \\
					&=2s^3(x_0+y_0+z_0)
				\end{align*}
			\end{item}
			\item $2s^3(x_0+y_0+z_0)/s^3=2(x_0+y_0+z_0)$. It is the divergence evaluated at point $(x_0,y_0,z_0)$.
			\begin{item}
				For II-14(b), $\textrm{div }\mathbf{F}=0$, so $\iint_S\mathbf{F}\cdot\hat{\mathbf{n}}\,dS=\iiint_V\nabla\cdot\mathbf{F}\,dV=0$.

				For II-14(c),
				\begin{align*}
					\iint_S\mathbf{F}\cdot\hat{\mathbf{n}}\,dS&=\iiint_V\nabla\cdot\mathbf{F}\,dV \\
					&=\int_{x_0-s/2}^{x_0+s/2}\int_{y_0-s/2}^{y_0+s/2}\int_{z_0-s/2}^{z_0+s/2}(-e^{-x}-e^{-y}-e^{-z})\,dz\,dy\,dx
				\end{align*}
				Using CAS system, this result is $-2(e^{-x_0}+e^{-y_0}+e^{-z_0})s^2\sinh(s/2)$. So this result devided by $s^3$ is $-2(e^{-x_0}+e^{-y_0}+e^{-z_0})\sinh(s/2)/s$.
				\[
					\lim_{s\rightarrow0}\frac{-2(e^{-x_0}+e^{-y_0}+e^{-z_0})\sinh(s/2)}{s}=0
				\]
			\end{item}
		\end{enumerate}
	}
\end{homeworkProblem}

%----------------------------------------------------------------------------------------
%	PROBLEM 9
%----------------------------------------------------------------------------------------

\begin{homeworkProblem}
	(II-16) (a) Calculate the divergence... (problem omitted)
	% Question

	\problemAnswer{
		\begin{enumerate}[label=(\alph*)]
			\item In Cartesian coordinates, $\textrm{div }\mathbf{F}=\nabla\cdot\mathbf{F}=\frac{\partial f(x)}{\partial x}+\frac{\partial f(y)}{\partial y}+\frac{\partial f(-2z)}{\partial z}=f'(x)+f'(y)-2f'(-2z)$. So at $(c,c,-c/2)$, $\textrm{div }\mathbf{F}=f'(c)+f'(c)-2f'(c)=0$.
			\item Obviously, each of the functions $f,g,h$ does not have dependence on the corresponding coordinate. so $\textrm{div }\mathbf{G}=0$.
		\end{enumerate}
	}
\end{homeworkProblem}

%----------------------------------------------------------------------------------------
%	PROBLEM 10
%----------------------------------------------------------------------------------------

\begin{homeworkProblem}
	(II-23) Verify the divergence theorem... (problem omitted)
	% Question

	\problemAnswer{
		\begin{enumerate}[label=(\alph*)]
			\begin{item}
				The surface integral could be devided into three parts, each at $b$ of the three coordinates.
				\[
					\iint_S\mathbf{F}\cdot\hat{\mathbf{n}}\,dS=\int_0^b\int_0^bb\,dy\,dz+\int_0^b\int_0^bb\,dx\,dz+\int_0^b\int_0^bb\,dx\,dy=3b^3
				\]
				\[
					\iiint_V\nabla\cdot\mathbf{F}\,dV=3\iiint_V\,dV=3b^3
				\]
			\end{item}
			\begin{item}
				The surface integral could be devided into two parts, the top of the cylinder, with unit normal vector $\hat{\mathbf{e}}_z$ and the outer curved surface, with unit normal vector $\hat{\mathbf{e}}_r$. The two surfaces on $xz$- and $yz$-plane is obviously zero.
				\[
					\iint_S\mathbf{F}\cdot\hat{\mathbf{n}}\,dS=\iint_{S_1}h\,dS+\iint_{S_2}R\,dS=\frac{1}{4}\pi R^2h+\frac{1}{4}\times2\pi R\times R=\frac{3\pi R^2h}{4}
				\]
				For this function in cylindrical coordinates,
				\[
					\textrm{div }\mathbf{F}=\frac{1}{r}\frac{\partial}{\partial r}(rF_r)+\frac{\partial F_z}{\partial z}=2+1=3
				\]
				\[
					\iiint_V\nabla\cdot\mathbf{F}\,dV=3\iiint_V\,dV=\frac{3\pi R^2h}{4}
				\]
			\end{item}
			\begin{item}
				\[
					\iint_S\mathbf{F}\cdot\hat{\mathbf{n}}\,dS=\iint_S R^2\,dS=4\pi R^4
				\]
				For this function in spherical coordinates,
				\[
					\textrm{div }\mathbf{F}=\frac{1}{r^2}\frac{\partial}{\partial r}(r^2F_r)=4r
				\]
				\[
					\iiint_V\nabla\cdot\mathbf{F}\,dV=4\iiint_V r\,dV=4\int_0^{2\pi}\int_0^\pi\int_0^R r\cdot r^2\sin\theta\,dr\,d\theta\,d\phi=4\pi R^4
				\]
			\end{item}
		\end{enumerate}
	}
\end{homeworkProblem}

%----------------------------------------------------------------------------------------

\end{document}
